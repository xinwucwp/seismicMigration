%\documentclass{cwpreport2012}
%\documentclass{article}
\documentclass[referee]{../../../texCls/mayWithTeaser}
%\setlength{\paperwidth}{8.5in}
%\setlength{\paperheight}{11.0in}
%\usepackage{times}
\usepackage{caption}
\usepackage{float}
\usepackage{wrapfig}
%\setcounter{totalnumber}{6}
\usepackage{xspace}
\usepackage{natbib}
\graphicspath{{./images/}}
\usepackage{caption}
\usepackage{float}
\usepackage{color}
\usepackage{wrapfig}
\usepackage{xspace}
\usepackage{natbib}
\usepackage{subfig}
\usepackage{graphicx}
\usepackage{amsmath}
\usepackage{amsbsy}
\usepackage{amssymb}
\usepackage{bm}
\usepackage{appendix}
\usepackage{bm}
\usepackage{xfrac}
\usepackage{placeins}
\usepackage{listings}
%\definecolor{lightgrey}{rgb}{0.9,0.9,0.9}
%\lstset{language=C++}
%\definecolor{dkgreen}{rgb}{0,0.6,0}
%\definecolor{gray}{rgb}{0.5,0.5,0.5}
%\definecolor{mauve}{rgb}{0.58,0,0.82}
\lstset{frame=tb,
  language=Java,
  aboveskip=3mm,
  belowskip=1mm,
  showstringspaces=false,
  columns=flexible,
  basicstyle={\ttfamily\small},
  numbers=none,
  numberstyle=\tiny\color{gray},
  keywordstyle=\color{blue},
  commentstyle=\color{dkgreen},
  stringstyle=\color{mauve},
  breaklines=true,
  breakatwhitespace=true
  tabsize=1
}
%\usepackage{lineno}
%\usepackage{appendix}
%\linenumbers
%\makeatletter
%\renewcommand{\thesubfigure}{\alph{subfigure}}
%\renewcommand{\@thesubfigure}{(\alph{subfigure})\hskip\subfiglabelskip}
\usepackage{boxedminipage}
\setlength{\topmargin}{-0.25in}
\setlength{\textheight}{8.75in}
\setlength{\textwidth}{6.5in}
\setlength{\oddsidemargin}{+.015625in}
\setlength{\evensidemargin}{+.015625in}

\begin{document}
\journal{HW2 for GPGN658: Seismic migration}
%\setcounter{page}{66}
\title{HW2 for GPGN658: Displacement from potentials}
\author[X.~Wu]
{Xinming Wu\\
CWID: 10622240\\
Center for Wave Phenomena, Colorado School of Mines, Golden, CO 80401, USA}
\maketitle
%%%%%%%%%%%%%%%%%%%%%%%%%%%%%%%%%%%%%%%%%%%%%%%%%%
\section{Potentials from a displacement field}
A displacement field $\mathbf{u(x},t)$ can be expressed as a sum of curl-free
and divergence-free forms by using the Helmholtz decomposition
\begin{equation}
  \mathbf{u} = \nabla\theta+\nabla\times\boldsymbol{\psi}.
\end{equation}
Given a displacement field $\mathbf{u}$, we can easily compute its corresponding
scalar potential
\begin{equation}
  \Theta=\nabla\cdot\mathbf{u}=\nabla\cdot(\nabla\theta),
\end{equation}
and vector potential
\begin{equation}
  \boldsymbol{\Psi}=\nabla\times\mathbf{u}=\nabla\times(\nabla\times\boldsymbol{\psi}).
\end{equation}
\section{Displacement from potentials}
Given the scalar and vector potentials $\Theta$ and $\boldsymbol{\Psi}$, we can
also reconstruct its corresponding displacement filed $\mathbf{u}$.\\
We know that the vector Laplacian of the displacement field $\mathbf{u}$ can be
expressed as
\begin{equation}
\begin{split}
  \nabla^2\mathbf{u}&=\nabla(\nabla\cdot\mathbf{u})-\nabla\times(\nabla\times\mathbf{u})\\
  &=\nabla\Theta-\nabla\times\boldsymbol{\Psi}.
\end{split}
\end{equation}
Both the left-hand and right-hand sides of the above equations are vectors.
Given the scalar and vector potentials $\Theta$ and $\boldsymbol{\Psi}$, we can
first compute the vector on the right-hand side, then we can compute all the
three components of the displacement field $\mathbf{u}$ by solving three Laplacian 
equations with a specific boundary condition.\\
Generally, such a Laplacian equation can be solved by using the Conjugate
Gradient method because a negative Laplacian operator is symmetric positive
definite. 
%%%%%%%%%%%%%%%%%%%%%%%%%%%%%%%%%%%%%%
\end{document}
