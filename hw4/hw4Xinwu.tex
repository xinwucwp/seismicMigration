%\documentclass{cwpreport2012}
%\documentclass{article}
\documentclass[referee]{../../../texCls/mayWithTeaser}
%\setlength{\paperwidth}{8.5in}
%\setlength{\paperheight}{11.0in}
%\usepackage{times}
\usepackage{caption}
\usepackage{float}
\usepackage{wrapfig}
%\setcounter{totalnumber}{6}
\usepackage{xspace}
\usepackage{natbib}
\graphicspath{{./images/}}
\usepackage{caption}
\usepackage{float}
\usepackage{color}
\usepackage{wrapfig}
\usepackage{xspace}
\usepackage{natbib}
\usepackage{subfig}
\usepackage{graphicx}
\usepackage{amsmath}
\usepackage{amsbsy}
\usepackage{amssymb}
\usepackage{bm}
\usepackage{appendix}
\usepackage{bm}
\usepackage{xfrac}
\usepackage{placeins}
\usepackage{listings}
%\definecolor{lightgrey}{rgb}{0.9,0.9,0.9}
%\lstset{language=C++}
%\definecolor{dkgreen}{rgb}{0,0.6,0}
%\definecolor{gray}{rgb}{0.5,0.5,0.5}
%\definecolor{mauve}{rgb}{0.58,0,0.82}
\lstset{frame=tb,
  language=Java,
  aboveskip=3mm,
  belowskip=1mm,
  showstringspaces=false,
  columns=flexible,
  basicstyle={\ttfamily\small},
  numbers=none,
  numberstyle=\tiny\color{gray},
  keywordstyle=\color{blue},
  commentstyle=\color{dkgreen},
  stringstyle=\color{mauve},
  breaklines=true,
  breakatwhitespace=true
  tabsize=1
}
%\usepackage{lineno}
%\usepackage{appendix}
%\linenumbers
%\makeatletter
%\renewcommand{\thesubfigure}{\alph{subfigure}}
%\renewcommand{\@thesubfigure}{(\alph{subfigure})\hskip\subfiglabelskip}
\usepackage{boxedminipage}
\setlength{\topmargin}{-0.25in}
\setlength{\textheight}{8.75in}
\setlength{\textwidth}{6.5in}
\setlength{\oddsidemargin}{+.015625in}
\setlength{\evensidemargin}{+.015625in}

\begin{document}
\journal{HW4 for GPGN658: Seismic migration}
%\setcounter{page}{66}
\title{HW4 for GPGN658: Finite-difference kernels}
\author[X.~Wu]
{Xinming Wu\\
CWID: 10622240\\
Center for Wave Phenomena, Colorado School of Mines, Golden, CO 80401, USA}
\maketitle
%%%%%%%%%%%%%%%%%%%%%%%%%%%%%%%%%%%%%%%%%%%%%%%%%%
\section{Second order}
Prove the stencil of the second derivative below has second-order accuracy.
\begin{equation}
\frac{\partial^2W}{\partial h^2}
\approx \frac{W_{h+(\triangle h)}-2W_h+W_{h-(\triangle h)}}{(\triangle h)^2}
\end{equation}
\textbf{Proof:}\\
From the Taylor's theorem, we have
\begin{equation}
W_{h+\triangle h} \approx W_h+\frac{1}{1!}\frac{\partial W}{\partial h}(\triangle h)
              +\frac{1}{2!}\frac{\partial^2 W}{\partial h^2}(\triangle h)^2
              +\frac{1}{3!}\frac{\partial^3 W}{\partial h^3}(\triangle h)^3
              +\frac{1}{4!}\frac{\partial^4 W}{\partial h^4}(\triangle h)^4
              +\cdots
\end{equation}
\begin{equation}
W_{h-\triangle h} \approx W_h-\frac{1}{1!}\frac{\partial W}{\partial h}(\triangle h)
              +\frac{1}{2!}\frac{\partial^2 W}{\partial h^2}(\triangle h)^2
              -\frac{1}{3!}\frac{\partial^3 W}{\partial h^3}(\triangle h)^3
              +\frac{1}{4!}\frac{\partial^4 W}{\partial h^4}(\triangle h)^4
              -\cdots
\end{equation}
From equations 2 and 3, we have
\begin{equation}
W_{h+\triangle h}+W_{h-\triangle h}\approx
2W_h+\frac{\partial^2 W}{\partial h^2}(\triangle h)^2
+\frac{1}{12}\frac{\partial^4 W}{\partial h^4}(\triangle h)^4+\cdots.
\end{equation}
Then we have
\begin{equation}
\frac{W_{h+\triangle h}-2W_h+W_{h-\triangle h}}{(\triangle h)^2}\approx
\frac{\partial^2 W}{\partial h^2}
+\frac{1}{12}\frac{\partial^4 W}{\partial h^4}(\triangle h)^2+\cdots.
\end{equation}
From equation 5, we have
\begin{equation}
\frac{\partial^2 W}{\partial h^2}\approx
\frac{W_{h+\triangle h}-2W_h+W_{h-\triangle h}}{(\triangle h)^2}
-\textcolor{red}{\frac{1}{12}\frac{\partial^4 W}{\partial h^4}(\triangle h)^2}-\cdots.
\end{equation}
Therefore, the finite-difference stencil in equation 1 has the second-order
accuracy.
%%%%%%%%%%%%%%%%%%%%%%%%%%%%%%%%%%%%%%
\section{Fourth order}
Prove the stencil of the second derivative below has fourth-order accuracy.
\begin{equation}
\frac{\partial^2W}{\partial h^2}
\approx \frac{-W_{h+2\triangle h}+16W_{h+\triangle h}-30W_h+
             16W_{h-\triangle h}-W_{h-2\triangle h}}{12(\triangle h)^2}
\end{equation}
According to the Taylor's theorem, we have
\begin{equation}
\begin{split}
W_{h+2\triangle h} 
\approx W_h
&+\frac{1}{1!}\frac{\partial W}{\partial h}(2\triangle h)
 +\frac{1}{2!}\frac{\partial^2 W}{\partial h^2}(2\triangle h)^2
 +\frac{1}{3!}\frac{\partial^3 W}{\partial h^3}(2\triangle h)^3\\
&+\frac{1}{4!}\frac{\partial^4 W}{\partial h^4}(2\triangle h)^4
 +\frac{1}{5!}\frac{\partial^5 W}{\partial h^5}(2\triangle h)^5
 +\frac{1}{6!}\frac{\partial^6 W}{\partial h^6}(2\triangle h)^6
 +\cdots
\end{split}
\end{equation}
\begin{equation}
\begin{split}
W_{h-2\triangle h} 
\approx W_h
&-\frac{1}{1!}\frac{\partial W}{\partial h}(2\triangle h)
 +\frac{1}{2!}\frac{\partial^2 W}{\partial h^2}(2\triangle h)^2
 -\frac{1}{3!}\frac{\partial^3 W}{\partial h^3}(2\triangle h)^3\\
&+\frac{1}{4!}\frac{\partial^4 W}{\partial h^4}(2\triangle h)^4
 -\frac{1}{5!}\frac{\partial^5 W}{\partial h^5}(2\triangle h)^5
 +\frac{1}{6!}\frac{\partial^6 W}{\partial h^6}(2\triangle h)^6
 -\cdots
\end{split}
\end{equation}
From equations 2, 3, 8 and 9, we have
\begin{equation}
-W_{h+2\triangle h}+16W_{h+\triangle h}-30W_h+
16W_{h-\triangle h}-W_{h-2\triangle h}\approx
12\frac{\partial^2W}{\partial h^2}(\triangle h)^2+
\frac{8}{3}\frac{\partial^6 W}{\partial h^6}(\triangle h)^6+\cdots
\end{equation}
Therefore, we have
\begin{equation}
\frac{\partial^2W}{\partial h^2}\approx
\frac{-W_{h+2\triangle h}+16W_{h+\triangle h}-30W_h+
16W_{h-\triangle h}-W_{h-2\triangle h}}{12(\triangle h)^2}
-\textcolor{red}{\frac{2}{9}\frac{\partial^6 W}{\partial h^6}(\triangle
h)^4}-\cdots,
\end{equation}
which means that the finite-difference stencil in equation 7 has the
fourth-order accuracy.
%%%%%%%%%%%%%%%%%%%%%%%%%%%%%%%%%%%%%%
\end{document}
