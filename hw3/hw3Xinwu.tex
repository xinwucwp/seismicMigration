%\documentclass{cwpreport2012}
%\documentclass{article}
\documentclass[referee]{../../../texCls/mayWithTeaser}
%\setlength{\paperwidth}{8.5in}
%\setlength{\paperheight}{11.0in}
%\usepackage{times}
\usepackage{caption}
\usepackage{float}
\usepackage{wrapfig}
%\setcounter{totalnumber}{6}
\usepackage{xspace}
\usepackage{natbib}
\graphicspath{{./images/}}
\usepackage{caption}
\usepackage{float}
\usepackage{color}
\usepackage{wrapfig}
\usepackage{xspace}
\usepackage{natbib}
\usepackage{subfig}
\usepackage{graphicx}
\usepackage{amsmath}
\usepackage{amsbsy}
\usepackage{amssymb}
\usepackage{bm}
\usepackage{appendix}
\usepackage{bm}
\usepackage{xfrac}
\usepackage{placeins}
\usepackage{listings}
%\definecolor{lightgrey}{rgb}{0.9,0.9,0.9}
%\lstset{language=C++}
%\definecolor{dkgreen}{rgb}{0,0.6,0}
%\definecolor{gray}{rgb}{0.5,0.5,0.5}
%\definecolor{mauve}{rgb}{0.58,0,0.82}
\lstset{frame=tb,
  language=Java,
  aboveskip=3mm,
  belowskip=1mm,
  showstringspaces=false,
  columns=flexible,
  basicstyle={\ttfamily\small},
  numbers=none,
  numberstyle=\tiny\color{gray},
  keywordstyle=\color{blue},
  commentstyle=\color{dkgreen},
  stringstyle=\color{mauve},
  breaklines=true,
  breakatwhitespace=true
  tabsize=1
}
%\usepackage{lineno}
%\usepackage{appendix}
%\linenumbers
%\makeatletter
%\renewcommand{\thesubfigure}{\alph{subfigure}}
%\renewcommand{\@thesubfigure}{(\alph{subfigure})\hskip\subfiglabelskip}
\usepackage{boxedminipage}
\setlength{\topmargin}{-0.25in}
\setlength{\textheight}{8.75in}
\setlength{\textwidth}{6.5in}
\setlength{\oddsidemargin}{+.015625in}
\setlength{\evensidemargin}{+.015625in}

\begin{document}
\journal{HW3 for GPGN658: Seismic migration}
%\setcounter{page}{66}
\title{HW3 for GPGN658: Dispersion relation}
\author[X.~Wu]
{Xinming Wu\\
CWID: 10622240\\
Center for Wave Phenomena, Colorado School of Mines, Golden, CO 80401, USA}
\maketitle
%%%%%%%%%%%%%%%%%%%%%%%%%%%%%%%%%%%%%%%%%%%%%%%%%%
\section{Dispersion relation for isotropic media}
In isotropic media, the constitutive law for stress and strain can be expressed
as
\begin{equation}
  t_{ij}=\lambda e_{kk}\delta_{ij}+2\mu e_{ij}.
\end{equation}
According to the equation of motions
\begin{equation}
  \rho\frac{\partial^2 u_i}{\partial t^2}=\frac{\partial t_{ij}}{\partial x_j},
\end{equation}
we have
\begin{equation}
  \rho\frac{\partial^2 u_i}{\partial t^2}
  =\frac{\partial \lambda e_{kk}\delta_{ij}+2\mu e_{ij}}{\partial x_j}
  =\lambda\delta_{ij}\frac{\partial e_{kk}}{\partial
  x_j}+2\mu\frac{e_{ij}}{\partial x_j}.
  \label{e:iso}
\end{equation}
We also have the geometric law below
\begin{equation}
  e_{kl}=\frac{1}{2}(\frac{\partial u_k}{\partial x_l}+\frac{\partial
  u_l}{\partial x_k}).
  \label{e:geoL}
\end{equation}
Substituting equation~\ref{e:geoL} into equation~\ref{e:iso}, and assuming
$\nabla\lambda\approx 0$, $\nabla\mu\approx 0$, we have
\begin{equation}
  \begin{split}
  \rho\frac{\partial^2 u_i}{\partial t^2}
  &=\lambda\delta_{ij}
   \frac{\partial^2 u_k}{\partial x_k\partial x_j}
   +\mu(
   \frac{\partial^2 u_i}{\partial x_j\partial x_j}+
   \frac{\partial^2 u_j}{\partial x_i\partial x_j})\\
  &=\lambda
   \frac{\partial^2 u_k}{\partial x_k\partial x_i}
   +\mu(
   \frac{\partial^2 u_i}{\partial x_j\partial x_j}+
   \frac{\partial^2 u_j}{\partial x_i\partial x_j}).
\end{split}
\end{equation}
Since $\frac{\partial^2 u_k}{\partial x_k\partial x_i}$ 
and $\frac{\partial^2 u_j}{\partial x_i\partial x_j}$ 
have the same structure, then we can combine them and obtain
\begin{equation}
  \rho\frac{\partial^2 u_i}{\partial t^2}=
  (\lambda+\mu)\frac{\partial^2 u_j}{\partial x_j\partial x_i}
   +\mu\frac{\partial^2 u_i}{\partial^2 x_j}.
\end{equation}
If we assume a planar wave $u_i=A_ie^{i(k_jx_j-\omega t)}$, and substitute it into the
above equation, we obtain
\begin{equation}
  \rho(-i\omega)^2u_i=(\lambda+\mu)[(ik_j)(ik_i)]u_j+\mu[(ik_j)^2]u_i,
\end{equation}
which can be further simplified as
\begin{equation}
  \rho \omega^2A_i=(\lambda+\mu)k_jk_iA_j+\mu k^2_jA_i.
\end{equation}
The equation above is the dispersion relation for isotropic media.
\section{Dispersion relation for VTI media}
As discussed in class, we the dispersion relation for general anisotropic media
as below
\begin{equation}
  C_{ijkl}k_jk_lA_k=\rho\omega^2A_i.
\end{equation}
If we let $n_j=\frac{k_j}{\omega s}$ and $n_l=\frac{k_l}{\omega s}$, where $s$
is slowness, then we have
\begin{equation}
  C_{ijkl}n_jn_lA_k=\rho v^2A_i.
  \label{e:dr}
\end{equation}
To further simplify the dispersion relation above, we define a $3\times 3$ matrix 
$\mathbf{G}$ with elements 
\begin{equation}
G_{ik}=C_{ijkl}n_jn_l.
\end{equation}
Then we obtain the below Christoffel equation from equation~\ref{e:dr}
\begin{equation}
  (G_{ik}-\rho v^2\delta_{ik})A_k=0.
\end{equation}
In the VTI media, the stiffness matrix is given by
\begin{equation}
  \left[\begin{matrix}
     C_{11} & C_{12} & C_{13} & 0 & 0 & 0\\
     C_{12} & C_{11} & C_{13} & 0 & 0 & 0\\
     C_{13} & C_{13} & C_{33} & 0 & 0 & 0\\
     0 & 0 & 0 & C_{55} & 0 & 0\\
     0 & 0 & 0 & 0 & C_{55} & 0\\
     0 & 0 & 0 & 0 & 0 & C_{66}\\
  \end{matrix}\right],
\end{equation}
where $C_{12}=C_{11}-2C_{66}$.
Then for the $3\times 3$ matrix $\mathbf{G}$, we have
\begin{equation}
  \begin{split}
    &G_{11}= C_{11}n^2_1+C_{66}n^2_2+C_{55}n^2_3\\
    &G_{22}= C_{66}n^2_1+C_{22}n^2_2+C_{55}n^2_3\\
    &G_{33}= C_{55}n^2_1+C_{55}n^2_2+C_{33}n^2_3\\
    &G_{21}=G_{12}= C_{12}n_1n_2+C_{66}n_2n_1=(C_{11}-C_{66})n_1n_2\\
    &G_{31}=G_{13}= C_{13}n_1n_3+C_{55}n_3n_1=(C_{13}+C_{55})n_1n_3\\
    &G_{32}=G_{23}= C_{23}n_2n_3+C_{55}n_3n_2=(C_{23}+C_{55})n_2n_3\\
  \end{split}
\end{equation}
Therefore, the dispersion relation for the VTI media can be expressed as
\begin{equation}
  \left[\begin{matrix}
      G_{11}-\rho v^2 & G_{12} & G_{13}\\
      G_{12} & G_{22}-\rho v^2 & G_{23}\\
      G_{13} & G_{23} & G_{33}-\rho v^2\\
  \end{matrix}\right]
  \left[\begin{matrix}
      A_1\\
      A_2\\
      A_3\\
  \end{matrix}\right]
  =
  \left[\begin{matrix}
      0\\
      0\\
      0\\
  \end{matrix}\right]
\end{equation}
\section*{Acknowledgments}
For the first part of isotropic media, I discussed with Tong Bai and Vladimir
Li. Both of them provided me valuable suggestions.
%%%%%%%%%%%%%%%%%%%%%%%%%%%%%%%%%%%%%%
\end{document}
